% Please do not change the document class
\documentclass{scrartcl}

% Please do not change these packages
\usepackage[hidelinks]{hyperref}
\usepackage[none]{hyphenat}
\usepackage{setspace}
\doublespace

% You may add additional packages here
\usepackage{amsmath}

% Please include a clear, concise, and descriptive title
\title{COMP230 - CPD Report}

% Please do not change the subtitle
\subtitle{COMP230 - CPD Report}

% Please put your student number in the author field
\author{1606119}

\begin{document}

\maketitle

\section{Introduction}
Many of the problems that I brought up in my reports last year have been resolved thanks to the plans that I constructed, however thanks to the wider variety of tasks and assignments that were given this study block, new challenges have appeared during this term that I have to plan how to overcome if I want to be a valuable team member and successful in my wanted career path, which has not changed since last year (Physics Programmer). A lot of the skills outlined in this report, much like last year, showed up during production time of the COMP230 BA game, however some are more general skills that I need to learn to be able to nail down in order to work effectively on solo projects, as well as in a group setting, as I feel both of these are applicable in the modern games industry. 


\section{Time Management for Programming Assessments}
The first skill that came to my attention was my lack of time management when it comes to programming work, upon reflection the plans that I made in my previous reports were too easily ignored and I consistently failed to follow them effectively and thus in this study block, I still fell behind on my work schedule and was working on projects right up to the hand-in date, which hindered my ability to create a final product up to the standard where the code is the level of quality that I wanted, such as having to not implement features that could get me a higher mark as I've not left myself enough time. However this was mostly self inflicted as I forgot to follow the work plan or simply put off the work, which I really should have learnt not to do by now as in a professional context this could cost me my job if I wasn't turning in work on time, or work that was not up to standard. To overcome this issue, I will undertake ``Timed'' work sessions everyday, where I work solid for a set period of time and then take a break, which is a method one of my lecturers informed me of, which I think will definitely be more achievable way of working as the work is more spread out and will seem less daunting, making myself much less likely to avoid doing or forgetting to do it. 

\section{Motivation to start a project}
The second skill that came up during the last study block was my lack of motivation at the start of a project, especially when it came to code heavy subjects such as COMP220, this I found was mostly due to my lack of knowledge about the language that we were going to be using, which in reflection I should have practised more over the summer break, which would have made the tasks much easier to understand and complete. The importance of this skill in the games development field is quite important in my opinion as a lack of motivation at the start of a project can definitely have an impact of the amount of work that I complete during the early sprints, which could also have an effect on my teams progress as my contribution might be integral to theirs getting completed, so it is definitely a skill that I nail quickly as if I continue to have this problem, the quality of my work will continue to suffer as I postpone work until it is no longer feasible, which definitely impacts the polish and amount of content in my final pieces. To overcome this issue, I will need to make better use of the facilities that I have at my disposal such as Pluralsite that can help me build up my knowledge that I require, and focus on getting the project that needs starting and working it to a point where the foundations are complete, as I have found that in the past, once a project is past this point I am much more motivated to continue work on it.

\section{Attend Regular Code Review Sessions}
The third skill that came to my attention is that I hardly go over my code with another party, which this study block has lead to some criticism from lecturers and team mates about certain aspects of my code, such as inconsistencies in my commenting, with this normally happening during set code review sessions that are near deadlines, leaving me little time to correct issues as I normally have to work to correct bigger issues, so I often struggle to meet the requirements for grades that I want to that relate to code quality. In a professional context I feel that this issue could lead to my code becoming hard for other programmers to understand, and thus harder to work with, which could definitely impact on the workflow of a project if I am always having to explain what my code is doing, due to issues that I haven't addressed. To address this issue I will hold more regular code review sessions, not just with my team mates but also with my lecturers as I feel having a more regular look at my code will allow me to gain a better understanding of where my common issues lie during the development process, and allow me more time to correct them so that I can obtain a higher grade in my assignments. 

\section{Knowledge and use of Unity}
 The fourth skill that came to up a lot during the last study block was my lack of knowledge when it came to tools that we were using for BA project, namely Unity and SVN, as this was the first time I've used Unity since college, I had forgotten a lot of the basics and thus struggled to work quickly enough to meet a lot of my targets during the development cycle, although I could have learnt these skills before hand, the choice of engine was undecided so I felt it better to build upon my Unreal skills instead as I use it more often. In a professional context I feel that having a limited knowledge of such a widely used engine would definitely hamper my ability to gain a job in a larger games company as having knowledge of more than 1 engine means I can work on a wider variety of projects. To address this issue, I will follow tutorials for Unity that are relevant to the project that I am working on such as for character movements and physics based interactions as this will allow me to work on a wider variety of mechanics within the project and make a larger overall contribution. 

\section{Better Communication with fellow Programmers}
The fifth skill that came to my attention was that I lack good communication with my fellow programmers, especially if I haven't worked with them before. This lack of skill normally resulted in me not understanding their work pattern and having a hard time keeping track of the progress they were making, which in a professional context really isn't good as I put off working on tasks until I know others have completed it, and without communicating with other programmers, I find that I find out long after it is complete and thus lose out on valuable development time , which could impact on the quality of the final product. To address this issue, I will make better use of the Slack channels in place so that I can communicate more often, and also check on the Trello board before I start work to see if anything has been updated so that I can lay out work that I can achieve. 

\section{Conclusion}
To conclude, the plans I have made in this report will hopefully make me a much more valuable asset to my development teams, and hopefully make me more employable in my future career as they cover a wide variety of areas that can make me a more effective and faster programmer.





\bibliographystyle{ieeetran}
\bibliography{references}

\end{document}
